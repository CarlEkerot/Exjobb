\documentclass[a4paper]{report}

\usepackage[round]{natbib}
\usepackage[nottoc,numbib]{tocbibind}
\usepackage{graphicx}
\usepackage[noline,noend,algoruled]{algorithm2e}
\usepackage{amssymb}
\usepackage{amsthm}
\usepackage[small]{caption}

\newtheorem*{definition}{Definition}

\begin{document}

\title{Automatic Structural Inference of Network Protocols}
\author{Fredrik Appelros \and Carl Ekerot}
\date{\today}
\maketitle

\begin{abstract}
An abstract is a brief summary of a research article, thesis, review,
conference proceeding or any in-depth analysis of a particular subject or
discipline, and is often used to help the reader quickly ascertain the paper's
purpose.
\end{abstract}

\tableofcontents

\chapter{Introduction}

\section{Background}
Here we will write a formal introduction to network protocols and why network
protocol inference is needed. Also explain what protocol inference is, of
course.

\section{Related work}
Here is a discussion about related work. We talk about different methodologies
along with their respective advantages and disadvantages. As an example here
is a citation \citep{cui07}.

\chapter{Theory}

\section{Features}
In machine learning, a feature is a measurable property of the input data. An
example of a feature when working with images could be the intensity of light
for a certain pixel. Features do not need to directly correspond to a
physically measurable property however, instead they can be constructed by
transforming other features. These new high-level features can then be used
just like any other feature in order to simplify a complex problem.

One common type of complex problem is working with high dimensional data. There
are different ways to deal with such data; one of them being to use a solving
method that scales linearly. Another way is to reduce the dimensionality of the
problem itself.

\subsection{Principal component analysis}
One method to reduce the dimensionality of a problem is to use principal
component analysis (PCA). PCA transforms the input data into another dataset
where each dimension accounts for as much variance as possible while requiring
it to maintain orthogonality to the other dimensions. The new dataset can be
constructed to contain fewer dimensions than the original data and will then
capture as much variance as possible in the given number of dimensions. PCA can
be accomplished through singular value decomposition of a matrix.

\section{Simple linear regression}
In statistics, a response variable $Y_i$ that is linearly dependent on an
explanatory variable $X_i$ can be modeled with a simple linear regression
model. The relationship between the two variables are then

\begin{equation}
    Y_i = \alpha + \beta X_i + \varepsilon_i
    \label{eq:slr}
\end{equation}

where $\alpha$ and $\beta$ are the \emph{regression coefficients} and
$\varepsilon_i$ is the error term. Given a set of data
$\displaystyle \{Y_i, X_i\}_1^n$ the regression coefficients can be estimated
through linear regression. This process is often called fitting as it can be
seen as trying to find the model with the best fit for the data. There are
different methods to do this as there is no consistent measure for what is the
best fit.

\subsection{Ordinary least squares}
Ordinary least squares (OLS) is a method for estimating the regression
coefficients in equation~\ref{eq:slr}. It does this by minimizing the sum of
squared residuals of the simple linear regression model. This can be formulated
as minimizing the following cost function.

\begin{equation}
    C(\alpha, \beta) = \sum_{i=1}^n(Y_i - \alpha - \beta X_i)^2
\end{equation}

The result of this is a model that minimizes the residuals globally across all
samples. This intrinsically makes the method \emph{non-robust}, i.e. sensitive
to outliers.

{
    \fontsize{10}{12}
    \selectfont
    \begin{algorithm}[t]
        \DontPrintSemicolon
        \BlankLine
        \KwIn{$D$, $M$, $n$, $\delta$, $\gamma$, $k$}
        \KwOut{$P_b$}
        \BlankLine
        $P_b \gets \varnothing$\;
        $\varepsilon_b \gets \infty$\;
        \For{$k$}{
            $I \gets n$ number of randomly chosen samples\;
            fit $M$ to $I$\;
            $I_m \gets$ samples where distance from $M \leq \delta$\;
            $I \gets I \cup I_m$\;
            \If{$|I| \geq \gamma$}{
                fit $M$ to $I$\;
                $\varepsilon \gets$ deviation from $M$ for all $I$\;
                \If{$\varepsilon < \varepsilon_b$}{
                    $P_b \gets$ parameters in $M$\;
                    $\varepsilon_b \gets \varepsilon$\;
                }
            }
        }
        \Return{$P_b$}
        \BlankLine
        \caption{RANSAC}
        \label{alg:ransac}
    \end{algorithm}
}

\subsection{RANSAC}
Random Sample Consensus (RANSAC) \citep{fischler81} is a \emph{robust} method
for fitting a dataset to a model and was originally intended for image
analysis. The robustness of the method allows it to avoid the influence of
outliers in the model. Given a dataset $D$, a model $M$, a set of inliers $I$
and two confidence parameters, $\delta$ and $\gamma$, it works the following
way:

\begin{enumerate}
    \item fit $M$ to $I$
    \item find the samples that are within $\delta$ distance from $M$
    \item add those samples to $I$
    \item continue if $|I|$ is at least $\gamma$
    \item fit $M$ to $I$
    \item evaluate $M$ from the residual of $I$
\end{enumerate}

One problem with this approach is that an initial set of inliers is required.
In most cases this is not available so we need to guess which samples might be
inliers. A simple solution to this is to take $n$ random samples and assume
that they are inliers. This only works if the assumption is actually true and
thus the resulting algorithm will be \emph{non-deterministic}, meaning that it
only produces a good fit with a certain probability. The algorithm is normally
run $k$ number of iterations until the probability of success is sufficiently
large. Pseudocode for RANSAC can be seen in algorithm~\ref{alg:ransac}.

The model that RANSAC operates on can be any kind of mathematical model, 
therefore when handling two-dimensional linear data the simple linear
regression model given in equation~\ref{eq:slr} can be used. This gives us a
method that solves the problem of simple linear regression but that is also
insensitive to outliers. A comparison between OLS and RANSAC on a dataset where
some samples have been replaced by random noise can be seen in
figure~\ref{fig:ransac}.

\begin{figure}[t]
    \includegraphics[width=\linewidth]{ransac}
    \caption{A comparison between OLS and RANSAC on a dataset containing outliers.}
    \label{fig:ransac}
\end{figure}

\section{Clustering}
A clustering is a grouping of samples based on similarity with respect to their
features. The features are predefined to represent distinct properties of the
samples. Similarities between samples are normally represented as distances in
an $n$-dimensional space, where $n$ is the number of features. Distances may be
calculated using any suitable norm.

In cluster analysis, the goal is normally to find a clustering from a
set of samples such that the membership of a cluster represents some true
relationship. In some clustering algorithms such as K-means \citep{macqueen67},
the number of clusters are predefined. When the number of clusters are unknown,
other algorithms such as DBSCAN, OPTICS and hierarchical clustering methods may
be used.

The clustering algorithm we use in our method is OPTICS, which is based on
DBSCAN. In the rest of this section we will explain these algorithms and
provide examples based on protocol data.

\subsection{DBSCAN}
Explain the basic idea of density-based clustering. Continue to describe the
DBSCAN algorithm as a practical implementation of the ideas introduced in the
section above. Conclude with an example based on the same DNS data as before.
Be sure to mention the problem with using the method on unknown data as the
parameters needs to be fine tuned.

DBSCAN is a \emph{density-based} clustering algorithm, as opposed to
partitioning clustering algorithms such as K-means. Density-based clustering 
algorithms provides the benefit of not having to provide an estimated number 
of clusters as a parameter to the algorithm.

DBSCAN was first presented by \citet{ester96}. The algorithm takes a set of 
samples $D$ and two parameters, $minPts$ and $\varepsilon$. $minPts$ decides
how many samples that are needed in order to form a cluster.

The densities are defined by the distances between a set of points.
The algorithm introduces the definition \emph{$\varepsilon$-neighborhood}
$N_{\varepsilon}(p)$ for a point $p$. The samples which are in 
$N_{\varepsilon}(p)$ are given by the following condition:

\begin{equation}
    N_{\varepsilon}(p) = \{ q \in D ~|~ dist(p,q) < \varepsilon  \}
    \label{eq:eps}
\end{equation}

The samples in $N_{\varepsilon}(p)$ are defined as \emph{directly density-
reachable} from $p$. Given that $|N_{\varepsilon}(p)| \ge minPts$, the samples
in $N_{\varepsilon}(p)$ forms a cluster. 

A cluster is not limited to containing samples which are direcly
density-reachable. The DBSCAN algorithm also defines that samples which are
not directly density-reachable may be \emph{density-reachable}.
A sample $q$ is density-reachable from a sample $p$ if there is a set of samples
$S = \{s_1, .., s_n\}$ where $p = s_1$, $q = s_n$ and $s_{i+1}$ is directly
density-reachable from $s_i$ for $0 < i < n$.

Since the density-reachable relationship is not symmetric, a looser
relationship is introduced: \emph{density-connected}. Two samples $p$ and
$q$ are density-connected if there exists a third point $o$ from which both
$p$ and $q$ are density-reachable. The density-connected relationship
is symmetric.

With these relationships, the algorithm defines cluster membership as
follows:

\begin{definition}
    The following conditions needs to be satisfied for a point $q$ to be a 
    member of a cluster $C$, given that there is a sample $p \in C$:
    \begin{enumerate}
        \item $q$ is density-reachable from $p$
        \item $q$ is density-connected to $p$
    \end{enumerate}
\end{definition}

One of the drawbacks of DBSCAN is that it can only find clusters with a density
higher than the density decided by the $\varepsilon$-paramter. It is also hard
to estimate the $\varepsilon$ and $minPts$-parameters for an unknown dataset.
This makes DBSCAN suitable for classifying data into known classes, but not as
suitable for finding underlying structures in entirely unknown data. In
figure~\ref{fig:dbscan} is an example of DBSCAN running on 5000 packets of DNS
data. The parameters are manually selected to $\varepsilon = 0.085$ and
$minPts = 200$.

\begin{figure}[H]
    \includegraphics[width=\linewidth]{dbscan_dns}
    \caption{An example of clustering with DBSCAN running on samples generated
    from 5000 packets of DNS data.}
    \label{fig:dbscan}
\end{figure}

\subsection{OPTICS}
Introduce the OPTICS algorithm which is an extension to the DBSCAN algorithm
that marries the concepts of hierarchical clustering with density-based
clustering. Emphasize the advantage of more robust parameters but also the
disadvantage of yet again introducing the problem of cluster extraction.

\subsubsection{Cluster extraction}
Explain that the problem with cluster extraction can be solved in OPTICS in
contrast to UPGMA. Start with the most basic threshold extraction and continue
with the more advanced hierarchical extraction method. Conclude with an example
of cluster extraction from DNS data.

\section{Sequence alignment}
Explain the concept of sequence alignment and why it is needed. (We need it
because of fields with variable length)

\subsection{Needleman-Wunsch}
Describe the Needleman-Wunsch algorithm for performing global sequence
alignment.

\section{State machines}
Explain what a state machine is and how it can be represented by an adjacency
matrix.

\chapter{Method}

\section{Approach}
Here we will write how we approached the problem of network protocol inference
and the different components that comprise our method. Start with explaining
why we need to cluster our data and continue to explain what needs to be done
after the clustering step to actually achieve protocol inference.

\section{Type inference}
Explain that our type inference step is actually two different clustering steps
performed in sequence.

\subsection{Initial clustering}
Explain our choice of features and clustering algorithm.

\subsection{Type distinguishers}
Describe our FD clustering algorithm.

\section{Field analysis}
Introduce the classes of fields that we have established as identifiable. Use
plots of their byte distributions to motivate our decision.

\subsection{Byte distribution}
Explain how we reached the decision to use byte distributions in our attempts
to find structures in protocols.

\subsection{Constant fields}

\subsection{Flag fields}

\subsection{Uniform fields}

\subsection{Number fields}

\subsection{Incremental fields}

\subsection{Length fields}
Explain how we use a linear model to model the relationship between length
fields and message lengths.

\section{Protocol state inference}
Explain how we build a state machine from our clusters.

\chapter{Results}

\section{Data sets}
Provide information about the data used to produce the results.

\section{Metrics}
Explain what metrics we use to measure our results.

\section{Clustering performance}
Display the results of applying our method to different data sets.

\section{Field inference performance}

\chapter{Discussion}

\section{Conclusions}
Draw conclusions about our method and compare it to the related methods.

\section{Limitations}
Give a short introduction to the different limitations of our method.

\subsection{Textual protocols}
Explain why our method is not applicable to textual protocols.

\subsection{Variable number of fields}
Explain the problem with protocols that contain a variable number of fields and
why our method does not accomodate for them. (Because we based our definition
of protocols on a fixed number of fields?)

\subsection{Non-aligned data}
Explain why we only try to find aligned fields and the problem with complexity
if we were to relax this requirement.

\subsection{Bit precision}
Talk a little bit about how some protocols do not use entire bytes as the sizes
of their fields and that we will not find exact boundaries for them.

\section{Future work}
Mention the ideas that has come up during the thesis work that we have not had
time to investigate further.

\subsection{Correspondence analysis}
Explain how correspondence analysis could give a better result than PCA.

\subsection{Timestamp identification}
Explain how timestamps could potentially be identified from their distinguished
byte distribution pattern.

\bibliographystyle{plainnat}
\bibliography{thesis}

\end{document}

