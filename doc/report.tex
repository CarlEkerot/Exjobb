\documentclass[a4paper]{report}
\usepackage[nottoc,numbib]{tocbibind}

\begin{document}
\title{Title}
\author{Fredrik Appelros \and Carl Ekerot}
\date{\today}
\maketitle

\begin{abstract}
An abstract is a brief summary of a research article, thesis, review, conference proceeding or any in-depth analysis of a particular subject or discipline, and is often used to help the reader quickly ascertain the paper's purpose.
\end{abstract}

\tableofcontents

\chapter{Introduction}
\section{Background}
\section{Related work}
Here is a discussion about related work. As an example here is a citation\cite{cui07}.
\chapter{Method}
\section{Approach}
\section{Clustering}
\subsection{Cluster theory}
\subsubsection{K-means - Example from SciKit-Learn}
\subsection{Features}
\subsubsection{High dimensional data}
\subsubsection{PCA - Example DNS traffic}
\subsection{Hierarchical clustering}
\subsubsection{UPGMA - Example DNS traffic (tree)}
\subsection{Density-based clustering}
\subsubsection{DBSCAN - Example DNS traffic}
\subsubsection{OPTICS}
\paragraph{Cluster extraction}
\section{Field analysis}
\subsection{Byte distribution}
\subsection{Field classes}
\subsection{RANSAC}
\subsection{Sequence alignment}
\section{State inference}
\chapter{Results}
\section{Data sets}
\chapter{Discussion}
\chapter{Conclusions}
\section{Limitations}
\section{Future work}
\subsection{Variable number of fields}
\subsection{Correspondence analysis}
\subsection{Bit precision}

\bibliographystyle{plain}
\bibliography{thesis}

\end{document}

